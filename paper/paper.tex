\documentclass[titlepage]{article}
\usepackage[T2A]{fontenc}
\usepackage[utf8]{inputenc}
\usepackage[russian]{babel}

\usepackage{amsmath}

\begin{document}

\begin{titlepage}

\begin{center}
	Министерство науки и высшего образования Российской Федерации\\
	Федеральное государственное автономное образовательное учреждение\\
	высшего образования\\
	\textbf{«Уральский федеральный университет\\
	имени первого президента России Б. Н. Ельцина»}
\end{center}

\vspace*{9em}{\centering\Huge
	Итоговая рассчётная работа по статистике\par}
\vspace{9em}

\begin{flushright}
	Дубовиков Н.Ю. \\ РИ-200003 \\ Преподователь Поторочина К.С.
\end{flushright}

\mbox{}
\vfill
\begin{center}
	Екатеринбург\\
	2021
\end{center}
\clearpage
\end{titlepage}

\tableofcontents
\newpage

\section{Введение}
\subsection{Предмет исследования}

Предметом данного иссследования является зависимость числа умышленных убийств (убийств первой степени
в законодательстве Соединённых Штатов) от количества владельцев оружия в разных городах США.

\subsection{Описание случайных величин}

Величины берутся в единицах на сто тысяч человек. Исходные данные были загружены с сайта ФБР. Данные актуальны на 2019 год.
\par
Количество умышленных убийств --- количество известных ФБР правонарушений, связанных с умышленным причинением смерти, произошедших
в 2019 году на территории определённого города.
\par
Количество владельцев оружия --- общее число граждан США, живущих на территории определённого города, получивших от ФБР разрешение
на покупку оружия. Определяется в основном строгостью законов в штате или городе, потому может сильно отличаться на разных
территориях страны. \par
Случайная выборка из ста значений исследуемых случайных величин представлена в таблице в приложении~\ref{sec:appendix} (названия городов
указаны как в оригинале --- на английском).

\subsection{Актуальность исследования}

Исследование позволяет установить, помогает ли ограничение оборота оружия снизить количество преступлений, связанных с насилием,
в частности убийств. Этим подтверждается её актуальность.

\subsection{Цель исследования}

Цель работы --- установить, есть ли зависимость между количеством умышленных убийств и количеством владельцев оружия на душу населения.

\section{Часть первая. Исследование одномерной выборки}

При исследовании одномерной выборки будем использовать величину X --- количество умышленных убийств на $100 000$ человек.
Объём выборки $N = 100$.

\subsection{Вариационный ряд}

Найдем из выборки минимальное и максимальное значение. $X_{min} = 0, X_{max} = 990$. Далее во формуле Стерджиса находим оптимальное количество интервалов:
\begin{equation*}
k = 1 + 3.332lg(N) \approx 7.664 \approx 8
\end{equation*}

Из размаха значений выборки и количества интервалов найдём длину одного интервала:
\begin{equation*}
h = \frac{X_{max} - X_{min}}{k} \approx 130
\end{equation*}

Теперь составим вариационный ряд:
\begin{table}[!ht]
    \centering
    \begin{tabular}{|l|l|l|l|l|l|l|l|l|}
    \hline
        $x_i$ & 0-130 & 130-260 & 260-390 & 390-520 & 520-650 & 650-780 & 780-910 & 910-1040 \\ \hline
        $n_i$ & 37 & 27 & 13 & 10 & 7 & 4 & 1 & 1 \\ \hline
    \end{tabular}
\end{table}

\clearpage
\appendix
\section{Выборка из собранных данных}
\label{sec:appendix}
\begin{table}[!ht]
    \centering
    \begin{tabular}{|l|l|l|}
    \hline
        Город & Количество убийств & Количество владельцев оружия \\ \hline
        Buckeye & 137 & 46300 \\ \hline
        Nevada & 255 & 48800 \\ \hline
        Dilley & 223 & 45700 \\ \hline
        GrossePointe & 117 & 40200 \\ \hline
        NewHaven & 297 & 44800 \\ \hline
        Harrisonville & 277 & 48800 \\ \hline
        Boston, & 607 & 14700 \\ \hline
        OakGrove & 207 & 48800 \\ \hline
        Tomahawk & 96 & 45300 \\ \hline
        Winchester & 481 & 51600 \\ \hline
        Hulbert & 171 & 54700 \\ \hline
        Emerson & 0 & 45200 \\ \hline
        Winneconne & 80 & 45300 \\ \hline
        VillaPark & 157 & 27800 \\ \hline
        BradentonBeach & 155 & 35300 \\ \hline
        Beatrice & 180 & 45200 \\ \hline
        Northglenn & 457 & 45100 \\ \hline
        Hutchinson & 378 & 48900 \\ \hline
        WarminsterTownship & 84 & 40700 \\ \hline
        Lyndhurst & 104 & 40000 \\ \hline
        Bath & 48 & 46800 \\ \hline
        Redmond & 102 & 42100 \\ \hline
        RobinsonTownship & 124 & 40700 \\ \hline
        Streetsboro & 121 & 40000 \\ \hline
        Chickasha & 433 & 54700 \\ \hline
        Murphy & 43 & 45700 \\ \hline
        Melbourne & 692 & 35300 \\ \hline
        Killeen & 384 & 45700 \\ \hline
        Maumee & 103 & 40000 \\ \hline
        Hanford & 449 & 28300 \\ \hline
        Pembroke & 92 & 44600 \\ \hline
        Elwood & 133 & 27800 \\ \hline
        Ashburnham & 126 & 14700 \\ \hline
        Athol & 402 & 14700 \\ \hline
        HedwigVillage & 224 & 45700 \\ \hline
        Elkton & 103 & 44600 \\ \hline
        Clarkdale & 68 & 46300 \\ \hline
        Auburn & 146 & 46800 \\ \hline
        Bokchito & 581 & 54700 \\ \hline
        Corning & 372 & 28300 \\ \hline
    \end{tabular}
\end{table}
\begin{table}[!ht]
    \centering
    \begin{tabular}{|l|l|l|}
    \hline
        Город & Количество убийств & Количество владельцев оружия \\ \hline
        Madison & 84 & 45200 \\ \hline
        Hastings & 246 & 40200 \\ \hline
        RanchoCordova & 297 & 28300 \\ \hline
        Viola & 0 & 27800 \\ \hline
        Jonesville & 272 & 40200 \\ \hline
        Ackerman & 139 & 55800 \\ \hline
        Mazon & 103 & 27800 \\ \hline
        SouthHaven & 555 & 40200 \\ \hline
        Fishers & 53 & 44800 \\ \hline
        Orion & 56 & 27800 \\ \hline
        Plymouth & 327 & 14700 \\ \hline
        Federalsburg & 528 & 30200 \\ \hline
        NewLondon & 294 & 23600 \\ \hline
        Niles & 774 & 40200 \\ \hline
        Rossville & 0 & 51600 \\ \hline
        LaVergne & 392 & 51600 \\ \hline
        Somerset & 0 & 40000 \\ \hline
        Emporia & 240 & 44600 \\ \hline
        Rosenberg & 380 & 45700 \\ \hline
        EagleVillage & 47 & 45300 \\ \hline
        Franklin & 165 & 51600 \\ \hline
        Ellensburg & 211 & 42100 \\ \hline
        Hemet & 398 & 28300 \\ \hline
        RomanForest & 148 & 45700 \\ \hline
        Perry & 245 & 54700 \\ \hline
        Refugio & 109 & 45700 \\ \hline
        Norfork & 0 & 57200 \\ \hline
        OliverSprings & 58 & 51600 \\ \hline
        Williamstown & 0 & 54600 \\ \hline
        Wynona & 0 & 54700 \\ \hline
        Dunwoody & 128 & 49200 \\ \hline
        Verona & 990 & 48800 \\ \hline
        Chicopee & 615 & 14700 \\ \hline
        GenevaTown & 40 & 45300 \\ \hline
        Avon & 63 & 42800 \\ \hline
        Auburn & 73 & 54600 \\ \hline
        Clinton & 160 & 54600 \\ \hline
        Mathis & 440 & 45700 \\ \hline
        Okmulgee & 615 & 54700 \\ \hline
        Newport & 806 & 50500 \\ \hline
        Halifax & 164 & 44600 \\ \hline
        Kirtland & 88 & 40000 \\ \hline
    \end{tabular}
\end{table}
\begin{table}[!ht]
    \centering
    \begin{tabular}{|l|l|l|}
    \hline
        Город & Количество убийств & Количество владельцев оружия \\ \hline
        Adrian & 689 & 40200 \\ \hline
        PalmSprings & 636 & 35300 \\ \hline
        FrontRoyal & 183 & 44600 \\ \hline
        MissouriCity & 141 & 45700 \\ \hline
        Visalia & 434 & 28300 \\ \hline
        PigeonForge & 736 & 51600 \\ \hline
        Beebe & 279 & 57200 \\ \hline
        Manhattan & 375 & 66300 \\ \hline
        Williamston & 176 & 40200 \\ \hline
        Berkeley & 503 & 28300 \\ \hline
        BerwynHeights & 214 & 30200 \\ \hline
        Muscatine & 197 & 43600 \\ \hline
        WilkinsTownship & 49 & 40700 \\ \hline
        LakeZurich & 40 & 27800 \\ \hline
        Merrillville & 144 & 44800 \\ \hline
        Buhl & 360 & 60100 \\ \hline
        PalosHeights & 40 & 27800 \\ \hline
        Martinez & 215 & 28300 \\ \hline
    \end{tabular}
\end{table}

\mbox{}
\vfill

\end{document}
